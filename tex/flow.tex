\documentclass{standalone}

\usepackage{tikz}
\usetikzlibrary{shapes,arrows}
\usepackage{amsmath,bm,times}
\newcommand{\mx}[1]{\mathbf{\bm{#1}}} % Matrix command
\newcommand{\vc}[1]{\mathbf{\bm{#1}}} % Vector command

\begin{document}
\pagestyle{empty}
	\centering

% We need layers to draw the block diagram
\pgfdeclarelayer{background}
\pgfdeclarelayer{foreground}
\pgfsetlayers{background,main,foreground}

% Define a few styles and constants
\tikzstyle{input}=[draw, fill=black!15!brown!10!, text width=8em, 
    text centered, minimum height=2.5em]
    
    
\tikzstyle{ann} = [above, text width=5em]

\tikzstyle{models} = [input, text width=6em, fill=black!20, rounded corners]

\tikzstyle{sub_models} = [input, text width=4em, fill=black!10]

\tikzstyle{function} = [input, text width=6em, fill=black!25!brown!20]

\tikzstyle{outputs} = [input, text width=9em, fill=brown!50]

\def\blockdist{2.3}

\def\external{7}

\def\edgedist{2.5}

\tikzset{>=latex}

\begin{tikzpicture}
    \node (model) [models, minimum height=12em,text depth=3.5cm] {xarray Dataset};
    
    \path (model.140)+(1.3,0) node () [] {\small{Coordinates}};
    
 
    \path[draw=black] (-1,1.3) -- (-1,0.8) -- (-0.5,0.8);
    \path[draw=black] (-0.8,1) -- (-1,0.8);
    
    \path (model.150)+(1.02,-0.6) node (sub_model) [sub_models] {};
    \path (model.150)+(1.08,-0.7) node (sub_model) [sub_models] {};
	\path (model.150)+(1.14,-0.8) node (sub_model) [sub_models] {};
    \path (model.150)+(1.2,-0.9) node (sub_model) [sub_models] {DataArray};
    
    \path (model.150)+(1.2,-2) node (sub_model) [sub_models, fill=black!10!brown!30!] {meta data};
    
   
   \path (model.150)+(-\external,1) node (raster_in) [input] {Raster data};
   \path (model.150)+(-\blockdist,1) node (raster) [function] {Affine transformation};
   \path [draw, ->] (raster_in) -- node [above, pos=0.2,] {\tiny geoTiff} (raster.west |- raster.west);
   \path [draw, ->] (raster) -- node [above] {\tiny{numpy}} (model.west |- raster.west);
    
  \path (model.150)+(\external,1) node (net_out) [input] {NetCDF};
   \path [draw, ->] (model.east |- net_out.west) -- node [above] {\tiny{nc}} (net_out.west |- net_out.west);
   
   \path (model.150)+(\external,0) node (fig_out) [input] {Figures};
  \path [draw, ->] (model.east |- fig_out.west) -- node [above] {\tiny{pdf / png}} (fig_out.west |- fig_out.west);
     
  \path (model.150)+(\external,-1) node (gis_out) [input] {GIS};
  \path [draw, ->] (model.east |- gis_out.west) -- node [above] {\tiny{GeoTIFF}} (gis_out.west |- gis_out.west);
     
    \path (model.150)+(\external,-2) node (morse_out) [input] {Morse};
    \path [draw, ->] (model.east |- morse_out.west) -- node [above] {\tiny{global png 3600px $\times$ 1800px}} (morse_out.west |- morse_out.west);
  
    
    
    \path (model)+(0,-3.2) node (attributes) [models, minimum height=2em] {Instance attributes};    
    \path [draw, <->] (model) -- node {} (model.south |- attributes.north);

    \path (model)+(0,3.4) node (python) [models, minimum height=2em] {Python};    
    \path [draw, <->] (model) -- node {} (model.north |- python.south);
   
   \path (model)+(0,4.6) node (python_api) [models, minimum height=2em] {Python api};    
	\path [draw, <->] (python_api) -- node {} (python_api.south |- python.north);
 
    \path (attributes)+(0,-1) node (grid_attr) [models, minimum height=2em] {Grid attributes};    
    
    \path (grid_attr.south west)+(-0.6,-0.8) node (class) {Python class Grid};
    
    
    
    \path (raster.south)+(0,-4.4) node (staal) [function] {Convolution of vector lines};
    \path (raster.south)+(-4.7,-4.4) node (staal_in) [input] {Vector data};
    \path [draw, ->] (staal_in) -- node [above, pos=0.3,] {\tiny e.g. shapefile} (staal.west |- staal.west);
    
    \path (model.150)+(-\blockdist,-5) node (staal) [function] {Interactive visualization};
    
    
    
    
   %  \draw [->] (model) -- node [attributes] {} + (\edgedist,0) 
    %node[right] {};

    
   % \draw [->] (model.20) -- node [ann] {pdf} + (\edgedist,0) 
    %    node[right] {Figure};
    %\draw [->] (model.-25) -- node [ann] {GeoTiff} + (\edgedist,0)
        %node [right] {GIS};
    %\draw [->] (model.-50) -- node [ann] {global png} + (\edgedist,0) 
     %   node[right] {Morse (ref)};
    
    \begin{pgfonlayer}{background}
        % Compute a few helper coordinates
        \path (raster.west |- raster.north)+(-0.3,0.3) node (a) {};
        \path (class.south -| model.east)+(+0.3,-0.3) node (b) {};        
        
        \path[fill=brown!30,rounded corners, draw=black!50]
            (a) rectangle (b) ;
            
        
   
            
          
        %\path (raster.north west)+(-.2,0.2) node (a) {};
        %\path (Import.south -| raster.east)+(+0.2,-0.2) node (b) {};
        
        
       % \path[fill=green!10,rounded corners, draw=black!50, dashed]
         %   (raster.west |- raster.north)+(-0.2,0.2) rectangle (raster.east -| raster.south)+(+0.2,+0.2);
            
       %\path[fill=blue!10,rounded corners, draw=black!50, dashed]
      %(staal.west  |- staal.north)+(-0.5,0.5) rectangle (staal.east  |- staal.south)+(-0.5,-0.5);
      
            
    \end{pgfonlayer}
\end{tikzpicture}



\end{document}
